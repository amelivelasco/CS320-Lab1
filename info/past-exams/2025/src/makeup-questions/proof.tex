\question[5]

Consider the grammar \(G\), with start symbol \(R\), representing the set of
palindromes over the alphabet \(\Sigma = \{a, b\}\):
%
\begin{align*}
    R &::= aRa \mid bRb \mid a \mid b \mid \epsilon
\end{align*}
%
From this, we construct the reflexive-transitive closure, the grammar \(G^*\)
with start symbol \(S\), representing \(L(G)^*\), by adding the single rule
%
\begin{align*}
    S &::= \epsilon \mid S R
\end{align*}

\begin{parts}
    \part \withpoints{1} Express the grammar \(G^*\) as an inductive relation
    defined by its production rules. 
    %
    \part \withpoints{4} Inductively prove that the relation \(G^*\) you defined
    contains \emph{every} word \(w \in \Sigma^*\), i.e., that \(G^*\) is
    universal.
\end{parts}

\begin{solutionorbox}[\fill]
    \addtolength{\jot}{1em}
    \withpoints{1}
    \(G^*\) as an inductive relation:
    \begin{gather*}
        \AxiomC{}
        \RightLabel{\(S_\epsilon\)}
        \UnaryInfC{\(\epsilon \in S\)}
        \DisplayProof
        \quad
        \AxiomC{\(w_1 \in S\)}
        \AxiomC{\(w_2 \in R\)}
        \RightLabel{\(S_R\)}
        \BinaryInfC{\(w_1 w_2 \in S\)}
        \DisplayProof
        \\
        \AxiomC{}
        \RightLabel{\(R_a\)}
        \UnaryInfC{\(a \in R\)}
        \DisplayProof
        \quad
        \AxiomC{}
        \RightLabel{\(R_b\)}
        \UnaryInfC{\(b \in R\)}
        \DisplayProof
        \quad
        \AxiomC{}
        \RightLabel{\(R_\epsilon\)}
        \UnaryInfC{\(\epsilon \in R\)}
        \DisplayProof
        \\
        \AxiomC{\(w \in R\)}
        \RightLabel{\(R_{aa}\)}
        \UnaryInfC{\(awa \in R\)}
        \DisplayProof
        \quad
        \AxiomC{\(w \in R\)}
        \RightLabel{\(R_{bb}\)}
        \UnaryInfC{\(bwb \in R\)}
        \DisplayProof
    \end{gather*}

    \withpoints{4}

    Essentially, the questions asks us to prove that every word \(w \in
    \Sigma^*\) can be decomposed into a sequence of palindromes. While this can
    be done by discovering larger palindromes, one may also note that every
    single letter is a palindrome. We can thus induct on the length of the word
    \(w\) as follows:
    
    \begin{description}
        \item[Base case:] \(|w| = 0\). Thus, \(w = \epsilon\). We have
        %
        \begin{gather*}
            \AxiomC{}
            \RightLabel{\(S_\epsilon\)}
            \UnaryInfC{\(\epsilon \in S\)}
            \DisplayProof
        \end{gather*}
        %
        \item[Inductive case:] \(|w| = n > 0\). The induction hypothesis states
        that for all \(k < n\), any word \(v \in \Sigma^*\) of length \(k\) is
        in \(S\). Since \(\Sigma = \{a, b\}\), \(w\) must end with either \(a\)
        or \(b\). Choose \(a\). The case for \(b\) is similar. Then, we can
        write \(w = va\) for some \(v \in \Sigma^*\) with length \(n - 1 < n\).
        By the induction hypothesis, there is a proof of \(v \in S\). We can
        then complete the proof for \(w \in S\) as:
        %
        \begin{gather*}
            \AxiomC{\(\ldots\)}
            \RightLabel{IH}
            \UnaryInfC{\(v \in S\)}
            \AxiomC{}
            \RightLabel{\(R_a\)}
            \UnaryInfC{\(a \in R\)}
            \RightLabel{\(S_R\)}
            \BinaryInfC{\(va \in S\)}
            \DisplayProof
        \end{gather*}
    \end{description}

    \pagebreak

    The induction proof is complete, and we have shown that for any word \(w \in
    \Sigma^*\), there exists a proof of \(w \in S\). Thus, \(\Sigma^* \subseteq
    L(G^*)\), and it follows that \(L(G^*) = \Sigma^*\). Hence, \(G^*\) is
    universal.

\end{solutionorbox}

\answerPage

