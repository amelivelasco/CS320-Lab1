\question[10]

Consider the following grammar that accepts well-formed propositional formulas
over the alphabet \(A = \{\land, \lor, \neg, ), (, atom\}\)
%
\begin{align*}
    S & ::= F ~\eof                                                \\
    F & ::= F \land F \mid F \lor F \mid \neg F \mid (F) \mid atom
\end{align*}

\begin{parts}
    \part \withpoints{1} Is the given grammar LL(1)? Justify your answer.
    %
    \part \withpoints{2} We use the term \emph{literal} to refer to either an
    atomic predicate \(atom\), or its negation \(\neg atom\). A \emph{clause} is
    a single literal \(l\) or a disjunction of literals (\(l_1 \lor l_2
    \lor \ldots \lor l_m\)). A formula is said to be in \emph{conjunctive
        normal form} (CNF) if it is a conjunction of clauses, i.e. \(C_1 \land C_2
    \land \ldots \land C_n\), where each \(C_i\) is a clause.

    Modify the grammar so it accepts only formulas that are in \emph{conjunctive
        normal form} (CNF).

    For example, the following formulas are in CNF, where \(a\) and \(b\) are atomic
    predicates:
    \begin{gather*}
        a, \quad
        a \land b, \quad
        (a \lor b), \quad
        \neg a \land \neg b, \quad
        (a \lor \neg c) \land b, \quad
        (\neg a \lor \neg b \lor c) \land (a \lor \neg b)
    \end{gather*}
    %
    \part \withpoints{1} Convert your grammar to be LL(1), if it is not already LL(1).
    %
    \part \withpoints{4} Draw the LL(1) parsing table for your grammar. Compute
    and show the \(\nullable\), \(\first\), and \(\follow\) sets for
    non-terminals in your grammar as intermediate steps.
    %
    \part \withpoints{2} Use the LL(1) parsing table to parse, or attempt to
    parse till error, the following string, showing intermediate state of the
    stack, and the chosen production at each step:
    \begin{gather*}
        (a \lor \neg c) \land b
    \end{gather*}

\end{parts}


\begin{center}
    \vfill
    \emph{
        Use the provided space on the \textbf{next three pages} for your answer.
    } 
    \vfill
\end{center}

\ifprintanswers
    \pagebreak
    \begin{solutionorbox}[\fill]
        \begin{parts}
            \part The grammar is not LL(1). \(F\) is left-recursive in multiple
            rules, and thus also shares a common prefix for multiple rules.
            \part A possible grammar for CNF formulas is:
            \begin{align*}
                S &::= F ~ \eof \\
                F &::= C ~F' \\
                F' &::= \land~ C ~F' \mid \epsilon \\
                C &::= L \mid (L \lor L~ C') \\
                C' &::= \lor~ L ~C' \mid \epsilon \\
                L &::= atom \mid \neg atom
            \end{align*}
            \part The grammar is already LL(1).
            \part \(\nullable\) can be computed as:
            %
            \begin{align*}
                \nullable(S) & = false \\
                \nullable(F) & = false \\
                \nullable(F') & = true \\
                \nullable(C) & = false \\
                \nullable(C') & = true \\
                \nullable(L) & = false
            \end{align*}
            %
            For \(\first\), the constraints are:
            %
            \begin{align*}
                \first(F) &\subseteq \first(S) \\
                \first(C) &\subseteq \first(F) \\
                \{\land\} &\subseteq \first(F')\\
                \first(L) &\subseteq \first(C) \\
                \{``("\} &\subseteq \first(C) \\
                \{\lor\} &\subseteq \first(C') \\
                \{atom, \neg\} &\subseteq \first(L)
            \end{align*}
            %
            which can be solved to get:
            \begin{align*}
                \first(S) &= \{atom, \neg, ``("\} \\
                \first(F) &= \{atom, \neg, ``("\} \\
                \first(F') &= \{\land\} \\
                \first(C) &= \{atom, \neg, ``("\} \\
                \first(C') &= \{\lor\} \\
                \first(L) &= \{atom, \neg\} 
            \end{align*}
            %
            For \(\follow\), the constraints are:
            \begin{align*}
                \{\eof\} &\subseteq \follow(F) \\
                \first(F') &\subseteq \follow(C) \\
                \follow(F) &\subseteq \follow(C) \\
                \follow(F) &\subseteq \follow(F') \\
                \first(F') &\subseteq \follow(C) \\
                \follow(F') &\subseteq \follow(C) \\
                \follow(C) &\subseteq \follow(L) \\
                \first(C') &\subseteq \follow(L) \\
                \{\lor, ``)"\} &\subseteq \follow(L) \\
                \{``)"\} &\subseteq \follow(C') \\
                \first(C') &\subseteq \follow(L)
            \end{align*}
            %
            which can be solved to get:
            \begin{align*}
                \follow(S) &= \{\} \\
                \follow(F) &= \{\eof\} \\
                \follow(F') &= \{\eof\} \\
                \follow(C) &= \{\land, \eof\} \\
                \follow(C') &= \{``)"\} \\
                \follow(L) &= \{\lor, ``)", \eof\}
            \end{align*}
            %
            We can finally write the parsing table:
            %
            \begin{center}
                \begin{tabular}{c c c c c c c c}
                    \hline
                    \textbf{State} & \(atom\) & \(\neg\) & \(\land\) & \(\lor\) & \((\) & \()\) & \(\eof\) \\
                    \hline
                    \(S\) & 1 & 1 & & & 1 & & \\
                    \(F\) & 1 & 1 & & & 1 & & \\
                    \(F'\) & & & 1 & & & & 2 \\
                    \(C\) & 1 & 1 & & & 1 & & \\
                    \(C'\) & & & & 1 & & 2 & \\
                    \(L\) & 1 & 2 & & & & & \\
                    \hline
                \end{tabular}
            \end{center}
            %
            There are no conflicts in the parsing table, additionally verifying
            that the grammar is LL(1).
            \pagebreak
            \part Parsing the string \((a \lor \neg c) \land b\): \\

            \begin{center}
                \begin{tabular}{c c c}
                    \hline
                    \textbf{Stack} & \textbf{Input} & \textbf{Next rule} \\
                    \hline
                    \(S\) & \((a \lor \neg c) \land b\) & \(S ::= F ~\eof\) \\
                    \(F ~\eof\) & \((a \lor \neg c) \land b\) & \(F ::= C ~F'\) \\
                    \(C ~F' ~\eof\) & \((a \lor \neg c) \land b\) & \(C ::= (L \lor L ~C')\) \\
                    \((L \lor L ~C') ~F' ~\eof\) & \((a \lor \neg c) \land b\) & match \(``("\) \\
                    \(L \lor L ~C') ~F' ~\eof\) & \(a \lor \neg c) \land b\) & \(L ::= atom\) \\
                    \(atom \lor L ~C') ~F' ~\eof\) & \(a \lor \neg c) \land b\) & match \(atom\) \\
                    \(\lor L ~C') ~F' ~\eof\) & \(\lor \neg c) \land b\) & match \(\lor\) \\
                    \(L ~C') ~F' ~\eof\) & \(\neg c) \land b\) & \(L ::= \neg atom\) \\
                    \(\neg atom ~C') ~F' ~\eof\) & \(\neg c) \land b\) & match \(\neg\) \\
                    \(atom ~C') ~F' ~\eof\) & \(c) \land b\) & match \(atom\) \\
                    \(C') ~F' ~\eof\) & \() \land b\) & \(C' ::= \epsilon\) \\
                    \() ~F' ~\eof\) & \() \land b\) & match \(``)"\) \\
                    \(F' ~\eof\) & \(\land b\) & \(F' ::= \land~C~F'\) \\
                    \(\land~C~F' ~\eof\) & \(\land b\) & match \(\land\) \\
                    \(C~F' ~\eof\) & \(\land b\) & match \(C ::= L\) \\
                    \(L~F' ~\eof\) & \(b\) & \(L ::= atom\) \\
                    \(atom~F' ~\eof\) & \(b\) & match \(atom\) \\
                    \(F' ~\eof\) & \(\eof\) & \(F' ::= \epsilon\) \\
                    \(\eof\) & \(\eof\) & match \(\eof\)\\
                    \hline
                \end{tabular}
            \end{center}
        \end{parts}
    \end{solutionorbox}
\fi

\answerPage
\answerPage
\answerPage

