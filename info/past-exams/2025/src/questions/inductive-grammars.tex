% inductive-grammars.tex

\question[5]

Consider a context-free language \(L\) over alphabet \(\Sigma\) defined by some
grammar \(G\), with start symbol \(S\). We define the language \(L'\) by the
following grammar \(G'\):
%
\begin{align*}
  R ::= RS \mid \epsilon  
\end{align*}
%
where \(R\) is the start symbol of \(G'\), and \(L(G') = L'\). 

\begin{parts}
  \part \withpoints{1} Express the grammar \(G'\) as a set of rules defining an inductive
  relation. You may assume that the inductive relation \(S \subseteq \Sigma^*\)
  has been defined. Note that a set is a special case of an inductive relation,
  having one argument.\\

  \SetTotalWidth
    \begin{solutionorbox}[10em]
      \addtolength{\jot}{1em}
      \begin{gather*}
        \AxiomC{\(w \in R\)}
        \RightLabel{\(G'_{start}\)}
        \UnaryInfC{\(w \in G'\)}
        \DisplayProof \\
        \AxiomC{\phantom{S}}
        \RightLabel{\(R_\epsilon\)}
        \UnaryInfC{\(\epsilon \in R\)}
        \DisplayProof \quad
        \AxiomC{\(w \in R\)}
        \AxiomC{\(v \in S\)}
        \RightLabel{\(R_s\)}
        \BinaryInfC{\(wv \in R\)}
        \DisplayProof
      \end{gather*}
    \end{solutionorbox}
  % \end{EnvFullwidth}
  \part \withpoints{4} Use these inductive definitions to prove that \(L' = L^*\). Use the fact
  that for any grammar \(G\) and word \(w\), \(w \in L(G) \iff w \in G\) where
  \(G\) is defined as an inductive relation.

  Recall that \(w \in L^*\) if and only if for some \(n \ge 0\), there exist
  \(w_1, \ldots, w_n\) such that \(w = w_1 \ldots w_n\) and \(w_i \in L\) for each
  \(0 < i \le n\).\\
\end{parts}

% \begin{EnvFullwidth}
  \SetTotalWidth
  \begin{solutionorbox}[\fill]
    Proving that \(L' = L^*\). We must show that \(w \in L'\) if and only
        if \(w \in L^*\). We will interchangeably use \(w \in L' \iff w \in G'
        \iff w \in R\), as well as \(w \in L \iff w \in G \iff w \in S\).
  
        \begin{enumerate}
          \item \withpoints{2} \(L' \subseteq L^*\): we need to show that if there is a
          derivation of \(w \in R\), then there exist \(w_1, \ldots, w_n\) such
          that \(w = w_1 \ldots w_n\) and \(w_i \in S\) for each \(0 < i \le n\).
  
          We prove this by induction on the depth of the derivation of \(w \in
          R\). 
  
          \begin{description}
            \item[Base case:] The proof is of depth 1. The only possible case is 
            the proof ends with \(R_\epsilon\), so \(w = \epsilon\):
            %
            \begin{gather*}
              \AxiomC{\phantom{S}}
              \RightLabel{\(R_\epsilon\)}
              \UnaryInfC{\(\epsilon \in R\)}
              \DisplayProof
            \end{gather*}
            %
            Choosing \(n = 0\), we have \(w = \epsilon \in L^*\) vacuously.
            \item[Inductive case:] The proof is of depth \(k \ge 2\). The
            inductive hypothesis states that for any \(v\), if there is a
            derivation of \(v \in R\) of depth \(< k\), then \(v \in L^*\). The
            last step of the proof must be \(R_s\), so we have:
            %
            \begin{gather*}
              \AxiomC{\ldots}
              \UnaryInfC{\(v_1 \in R\)}
              \AxiomC{\ldots}
              \UnaryInfC{\(v_2 \in S\)}
              \RightLabel{\(R_s\)}
              \BinaryInfC{\(v_1v_2 \in R\)}
              \DisplayProof
            \end{gather*}
            %
            where \(w = v_1v_2\). Given that we have a derivation of \(v_1 \in R\)
            with depth \(< k\), from the inductive hypothesis, we know that there
            exists an \(m \ge 0\) such that \(v_1 = u_{1} \ldots u_{m}\) and there
            is a derivation of \(u_{i} \in S\) for each \(0 < i \le m\). Thus, \(w
            = u_1\ldots u_m v_2\). But, we also have a derivation for \(v_2 \in
            S\). Finally, choose \(n = m + 1\), and choose \(w_1 = u_1, \ldots,
            w_m = u_m,\) and \(w_{m+1} = v_2\). We have \(w \in L^*\) again.
          \end{description}
  
          \item \withpoints{2} \(L^* \subseteq L'\): we need to show that if there
          exist \(w_1, \ldots, w_n\) such that \(w = w_1 \ldots w_n\) and there
          is a derivation of \(w_i \in L\) for each \(0 < i \le n\), then there
          is a derivation of \(w \in L'\).
  
          We prove this by induction on \(n\).
  
          \begin{description}
            \item[Base case:] \(n = 0\). This means that \(w = \epsilon\). We can
            construct a proof for \(w \in R\):
            %
            \begin{gather*}
              \AxiomC{\phantom{S}}
              \RightLabel{\(R_\epsilon\)}
              \UnaryInfC{\(\epsilon \in R\)}
              \DisplayProof
            \end{gather*}
            %
            \item[Inductive case:] \(n = k + 1\), there exist words \(w_1, \ldots,
            w_{k+1}\) such that \(w = w_1\ldots w_{k+1}\) and there is a
            derivation of \(w_i \in S\) for each \(0 < i \le k + 1\). The
            induction hypothesis states that for any word \(v \in L^*\) with a
            decomposition of \(\le k\) words in \(L\), there exists a derivation
            of \(v \in R\). Consider the word \(w_1\ldots w_{k}\). By the
            induction hypothesis, there exists a derivation of \(w_1\ldots w_{k}
            \in R\). We can construct a proof for \(w \in R\):
            \begin{gather*}
              \AxiomC{\ldots}
              \UnaryInfC{\(w_1\ldots w_{k} \in R\)}
              \AxiomC{\ldots}
              \UnaryInfC{\(w_{k+1} \in S\)}
              \RightLabel{\(R_s\)}
              \BinaryInfC{\(w_1\ldots w_{k} w_{k+1} \in R\)}
              \DisplayProof
            \end{gather*}
          \end{description}
        Thus, the proof is complete, and \(L' = L^*\).
        \end{enumerate}
  \end{solutionorbox}

  \ifprintanswers
    %%
  \else
    \newpage
    \SetTotalWidth
    \begin{solutionorbox}[\fill]
      %
    \end{solutionorbox}
  \fi
% \end{EnvFullwidth}
