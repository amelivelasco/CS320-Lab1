% grammar-transformation.tex

\newcommand{\alignrightstretch}{6cm}

\question[5]
\label{q:ll1-transform}

Consider the following grammar for a language consisting of variables,
constructors, and match-case statements:
%
\begin{align}
  S &::= Expr ~\eof \tag{0} \\
  Expr &::= SimpleExpr ~Expr' \\
  Expr' &::= \epsilon \mid Match \\
  SimpleExpr &::= \lstinline|var| \mid Cons \mid (Expr) \\
  Cons &::= \lstinline|id| (ExprList) \\
  ExprList &::= \epsilon \mid NExprList \\
  NExprList &::= Expr \mid Expr, ~NExprList \\
  Match &::= \lstinline|match|~ CaseList \\
  CaseList &::= Case \mid Case ~CaseList \\
  Case &::= \lstinline|case| ~SimpleExpr ~\lstinline|=>| ~Expr
  %% Computer readable one to match sets at
  %% https://www.cs.princeton.edu/courses/archive/spring20/cos320/LL1/
  % Expr ::= SimpleExpr Expr' 
  % Expr' ::= '' 
  % Expr' ::= Match 
  % SimpleExpr ::= 'var'
  % SimpleExpr ::= Cons 
  % SimpleExpr ::= '(' Expr ')' 
  % Cons ::= 'id' '(' ExprList ')' 
  % ExprList ::= '' 
  % ExprList ::= NExprList 
  % NExprList ::= Expr 
  % NExprList ::= Expr ',' NExprList 
  % Match ::= 'match' CaseList 
  % CaseList ::= Case 
  % CaseList ::= Case CaseList 
  % Case ::= 'case' SimpleExpr '=>' Expr
\end{align}

where \lstinline|var|, \lstinline|id|, \lstinline|match|, \lstinline|case|,
\lstinline|=>|, \lstinline|(|, \lstinline|)|, \lstinline|,|, and \eof{} are all
terminal tokens.

\vfill
\begin{center}
  \emph{This question has four (4) subparts, one on each of the following pages.}
\end{center}
\vfill

\begin{parts}
  \pagebreak
  \part \withpoints{1} Compute \(\nullable\) for each non-terminal in the grammar. \\
    \SetTotalWidth
    \begin{tcolorbox}[
      width=\dimexpr\linewidth - \leftmargin\relax, 
      left skip=\dimexpr-\leftmargin\relax,
      height fill,
    ]
      \ifprintanswers
      \else
        \addtolength{\jot}{1em}
      \fi
      \begin{align*}
      \nullable(S) &= \ifAns{false} \\
      \nullable(Expr) &= \ifAns{false} \\
      \nullable(Expr') &= \ifAns{true} \\
      \nullable(SimpleExpr) &= \ifAns{false} \\
      \nullable(Cons) &= \ifAns{false} \\
      \nullable(ExprList) &= \ifAns{true} \\
      \nullable(NExprList) &= \ifAns{false} \\
      \nullable(Match) &= \ifAns{false} \\
      \nullable(CaseList) &= \ifAns{false} \\
      \nullable(Case) &= \ifAns{false} \\
      \ifprintanswers
      \else
      &\hspace*{\alignrightstretch}
      \fi
      \end{align*}
    \end{tcolorbox}
  
    \pagebreak
  \part \withpoints{1} Compute the \(\first(\cdot)\) sets for each non-terminal in the grammar. \\
    \SetTotalWidth
    \begin{tcolorbox}[
      width=\dimexpr\linewidth - \leftmargin\relax, 
      left skip=\dimexpr-\leftmargin\relax,
      height fill
    ]
      \ifprintanswers
        Constraints for \(\first\) sets, labelled by which rule number they come
        from:
        \begin{align*}
          \first(S) &\subseteq \first(Expr) \tag{0} \\
          %
          \first(SimpleExpr) &\subseteq \first(Expr) \tag{1} \\
          %
          \first(Match) &\subseteq \first(Expr') \tag{2} \\
          %
          \{(, \lstinline|var|\} &\subseteq \first(SimpleExpr) \tag{3} \\
          \first(Cons) &\subseteq \first(SimpleExpr) \tag{3} \\
          %
          \{\lstinline|id|\} &\subseteq \first(Cons) \tag{4} \\
          \first(NExprList) &\subseteq \first(ExprList) \tag{5} \\
          \first(Expr) &\subseteq \first(NExprList) \tag{6} \\
          \{\lstinline|match|\} &\subseteq \first(Match) \tag{7} \\
          \first(Case) &\subseteq \first(CaseList) \tag{8} \\
          \{\lstinline|case|\} &\subseteq \first(Case) \tag{9}
        \end{align*}
        %
        Which can be solved to get:
      \else
        \addtolength{\jot}{1em}
      \fi
      \begin{align*}
      \first(S) &= \ifAns{\{(, \lstinline|var|, \lstinline|id|\}} \\
      \first(Expr) &= \ifAns{\{(, \lstinline|var|, \lstinline|id|\}} \\
      \first(Expr') &= \ifAns{\{\lstinline|match|\}} \\
      \first(SimpleExpr) &= \ifAns{\{(, \lstinline|var|, \lstinline|id|\}} \\
      \first(Cons) &= \ifAns{\{\lstinline|id|\}} \\
      \first(ExprList) &= \ifAns{\{(, \lstinline|var|, \lstinline|id|\}} \\
      \first(NExprList) &= \ifAns{\{(, \lstinline|var|, \lstinline|id|\}} \\
      \first(Match) &= \ifAns{\{\lstinline|match|\}} \\
      \first(CaseList) &= \ifAns{\{\lstinline|case|\}} \\
      \first(Case) &= \ifAns{\{\lstinline|case|\}} \\
      \ifprintanswers
      \else
      &\hspace*{\alignrightstretch}
      \fi
      \end{align*}
    \end{tcolorbox}
  \pagebreak
  \part \withpoints{1} Compute the \(\follow(\cdot)\) sets for each non-terminal except \(S\) in the
  grammar. \\
    \SetTotalWidth
    \begin{tcolorbox}[
      width=\dimexpr\linewidth - \leftmargin\relax, 
      left skip=\dimexpr-\leftmargin\relax,
      height fill,
    ]
      %
      \ifprintanswers
      Constraints for \(\follow\) sets, labelled by which rule number they arise
      from:
      \begin{align*}
        \{\eof\} &\subseteq \follow(Expr) \tag{0} \\
        %
        \first(Expr') &\subseteq \follow(SimpleExpr) \tag{1} \\
        \follow(Expr) &\subseteq \follow(SimpleExpr) \tag{1} \\
        \follow(Expr) &\subseteq \follow(Expr') \tag{1} \\
        %
        \follow(Expr') &\subseteq \follow(Match) \tag{2} \\
        %
        \follow(SimpleExpr) &\subseteq \follow(Cons) \tag{3} \\
        \{)\} &\subseteq \follow(Expr) \tag{3} \\ 
        %
        \{)\} &\subseteq \follow(ExprList) \tag{4} \\
        %
        \follow(ExprList) &\subseteq \follow(NExprList) \tag{5} \\
        %
        \{,\} &\subseteq \follow(Expr) \tag{6} \\
        %
        \follow(Match) &\subseteq \follow(CaseList) \tag{7} \\
        %
        \follow(CaseList) &\subseteq \follow(Case) \tag{8} \\
        \first(CaseList) &\subseteq \follow(Case) \tag{8} \\
        %
        \{\lstinline|=>|\} &\subseteq \follow(SimpleExpr) \tag{9} \\
        \follow(Case) &\subseteq \follow(Expr) \tag{9}
      \end{align*}
      %
      which can be solved to get
      %
      \else
        \addtolength{\jot}{1em}
      \fi
      \begin{align*}
        \follow(Expr) &= \ifAns{\{\lstinline|case|, ``,", ``)", \eof\}} \\
        \follow(Expr') &= \ifAns{\{\lstinline|case|, ``,", ``)", \eof\}} \\
        \follow(SimpleExpr) &= \ifAns{\{\lstinline|match|, \lstinline|case|, \lstinline|=>|, ``,", ``)", \eof\}} \\
        \follow(Cons) &= \ifAns{\{\lstinline|match|, \lstinline|case|, \lstinline|=>|, ``,", ``)", \eof\}} \\
        \follow(ExprList) &= \ifAns{\{)\}} \\
        \follow(NExprList) &= \ifAns{\{)\}} \\
        \follow(Match) &= \ifAns{\{\lstinline|case|, ``,", ``)", \eof\}} \\
        \follow(CaseList) &= \ifAns{\{\lstinline|case|, ``,", ``)", \eof\}} \\
        \follow(Case) &= \ifAns{\{\lstinline|case|, ``,", ``)", \eof\}} \\
        \ifprintanswers
        \else
        &\hspace*{\alignrightstretch}
        \fi
      \end{align*}
    \end{tcolorbox}

  \pagebreak
  \part \withpoints{2} Construct the parsing table for this grammar. Use the production options
  in order as given to fill in the parse table. For example, with \(Expr' ::=
  \epsilon \mid Match\), write \(1\) for choosing the rule \(Expr' ::=
  \epsilon\), and \(2\) for choosing \(Expr' ::= Match\).
  
  Show that the grammar is \emph{not} LL(1) by marking every conflict in the
  table.\\

  \SetTotalWidth
  \begin{tcolorbox}[
      width=\dimexpr\linewidth - \leftmargin\relax, 
      left skip=\dimexpr-\leftmargin\relax,
      height fill,
    ]
      \begin{center}
        \renewcommand{\arraystretch}{1.5}
        \begin{tabular}{| c | c | c | c | c | c | c | c | c | c |}
          \hline
          \textbf{Non-terminal} & \lstinline|var| & \lstinline|id| & \lstinline|(|  & \lstinline|)| & \lstinline|match| & \lstinline|case| & \lstinline|=>| & \lstinline|,| & \eof \\ 
          \hline
          %% PHANTOMS ARE A HACK TO MAKE COLUMNS EQUAL SIZED
          % var id ( ) match case => , eof
          Expr & \ifAns{1} & \ifAns{1} & \ifAns{1} & & & & & & \\
          \hline
          Expr' & & & & \ifAns{1} & \ifAns{2} & \ifAns{1} & & \ifAns{1} & \ifAns{1} \\
          \hline
          SimpleExpr & \ifAns{1} & \ifAns{2} & \ifAns{3} & & & & & & \\
          \hline
          Cons & & \ifAns{1} & & & & & & & \\
          \hline
          ExprList & \ifAns{2} & \ifAns{2} & \ifAns{2} & \ifAns{1} & & & & & \\
          \hline
          NExprList & \ifAns{\textbf{\textcolor{red}{1, 2}}} & \ifAns{\textbf{\textcolor{red}{1, 2}}}& \ifAns{\textbf{\textcolor{red}{1, 2}}}& & & & & & \\
          \hline
          Match & & & & & \ifAns{1} & & & & \\
          \hline
          CaseList & & & & & & \ifAns{\textbf{\textcolor{red}{1, 2}}} & & & \\
          \hline
          Case & \phantom{1, 2} & \phantom{1, 2} & \phantom{1, 2} & \phantom{1, 2} & \phantom{1, 2} & \ifAns{1} & \phantom{1, 2} & \phantom{1, 2} & \phantom{1, 2} \\
          \hline
        \end{tabular}
      \end{center}
    \end{tcolorbox}
\end{parts}
