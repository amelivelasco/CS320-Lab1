
\usepackage{amsmath, amssymb, bm}
\usepackage{tikz}
\usetikzlibrary{arrows, shapes, positioning, calc, automata}
\usepackage{bussproofs}
\usepackage{multicol}
\usepackage{ifdraft}
\usepackage[normalem]{ulem}
\usepackage{etoolbox}
\usepackage{forest}
\usepackage{hyperref}
\usepackage{tabularx}
\usepackage{pgf}
\usepackage{xspace}
\usepackage{qrcode}
\usepackage{exam-randomizechoices}

% modifying exam
% \renewcommand{\questionlabel}{\textbf{Question \thequestion.}}
\qformat{\textbf{Question \thequestion}.\hfill(\thepoints)}
\renewcommand{\thepartno}{\roman{partno}}

% use to write questions where partial information is given to students, the
% \ifAns parts will be hidden in the students' version
\newcommand{\ifAns}[1]{
    \ifprintanswers
        #1
    \else
        \phantom{#1}
    \fi
}

% exam/solution boxes, sometimes
\usepackage[breakable]{tcolorbox}
\tcbset{
  colback = white,
  colframe = black,
  sharp corners = all,
  boxrule = 0.5pt % default thickness of hrule
}

\usepackage{listings}

\definecolor{dkgreen}{rgb}{0,0.6,0}
\definecolor{gray}{rgb}{0.5,0.5,0.5}
\definecolor{mauve}{rgb}{0.58,0,0.82}

\lstdefinestyle{scalaStyle}{
  frame=tb,
%   language=scala,
  aboveskip=3mm,
  belowskip=3mm,
  showstringspaces=false,
  columns=flexible,
  basicstyle=\small\ttfamily,
  numbers=none,
  numberstyle=\tiny\color{gray},
  keywordstyle=\color{blue},
  commentstyle=\color{dkgreen},
  stringstyle=\color{mauve},
  frame=none,
  breaklines=true,
  breakatwhitespace=true,
  tabsize=2,
}

\lstset{style=scalaStyle}

% lstinline in math mode
% https://tex.stackexchange.com/a/127018

\usepackage{letltxmacro}
\newcommand*{\SavedLstInline}{}
\LetLtxMacro\SavedLstInline\lstinline
\DeclareRobustCommand*{\lstinline}{%
  \ifmmode
    \let\SavedBGroup\bgroup
    \def\bgroup{%
      \let\bgroup\SavedBGroup
      \hbox\bgroup
    }%
  \fi
  \SavedLstInline
}

% compute the \fill length manually
\newcommand\measurepage{\dimexpr\pagegoal-\pagetotal-\baselineskip\relax}

% https://tex.stackexchange.com/a/89254
% full length solution splitting
\makeatletter
\def\SetTotalWidth{\advance\linewidth by \@totalleftmargin
\@totalleftmargin=0pt}
\makeatother

\newcommand{\answerPage}{
  \ifprintanswers
      %
  \else
      \pagebreak
      \begin{solutionorbox}[\fill] \end{solutionorbox}
  \fi
}

\newcommand{\todo}[1]{{[TODO: \color{red} #1]}}

\newcommand{\naturals}{\mathbb{N}}

\DeclareMathOperator{\nullable}{nullable}
\DeclareMathOperator{\first}{first}
\DeclareMathOperator{\follow}{follow}
\newcommand{\eof}{\ensuremath{\mathbf{EOF}}}

% time HH:MM
\newcommand{\formatTime}[2]{#1:#2}

% starting time: HH:MM, and added minutes MM
\newcommand{\addTime}[3]{%
  \pgfmathsetmacro{\startH}{#1}%
  \pgfmathsetmacro{\startM}{#2}%
  \pgfmathsetmacro{\addM}{#3}%
  \pgfmathsetmacro{\totalminutes}{\startH * 60 + \startM + \addM}%
  \pgfmathsetmacro{\adjustedminutes}{mod(\totalminutes, 24*60)}%
  \pgfmathsetmacro{\newHour}{int(\adjustedminutes / 60)}%
  \pgfmathsetmacro{\newMinute}{int(mod(\adjustedminutes, 60))}%
  \formatTime{\newHour}{\newMinute}%
}

\newcommand{\withpoints}[1]{\fbox{#1 points}}

\newcommand{\lexChoice}[2]{\textbf{#1}: \begin{aligned} #2 \end{aligned}}
\newcommand{\setChoice}[2]{\fbox{\(\lexChoice{#1}{#2}\)}}

\newcommand{\falseChoice}{
  \\
  \begin{oneparchoices}
    \choice True
    \CorrectChoice False
  \end{oneparchoices}
}
\newcommand{\trueChoice}{
  \\
  \begin{oneparchoices}
    \CorrectChoice True
    \choice False
  \end{oneparchoices}
}
